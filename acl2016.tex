%
% File acl2016.tex
%
%% Based on the style files for ACL-2015, with some improvements
%%  taken from the NAACL-2016 style
%% Based on the style files for ACL-2014, which were, in turn,
%% Based on the style files for ACL-2013, which were, in turn,
%% Based on the style files for ACL-2012, which were, in turn,
%% based on the style files for ACL-2011, which were, in turn,
%% based on the style files for ACL-2010, which were, in turn,
%% based on the style files for ACL-IJCNLP-2009, which were, in turn,
%% based on the style files for EACL-2009 and IJCNLP-2008...

%% Based on the style files for EACL 2006 by
%%e.agirre@ehu.es or Sergi.Balari@uab.es
%% and that of ACL 08 by Joakim Nivre and Noah Smith

\documentclass[11pt]{article}
\usepackage{acl2016}
\usepackage{times}
\usepackage{url}
\usepackage{latexsym}
\usepackage[normal]{caption}
\usepackage{subcaption}
\usepackage{epsfig}
\usepackage{graphicx}

%\aclfinalcopy % Uncomment this line for the final submission
%\def\aclpaperid{***} %  Enter the acl Paper ID here

%\setlength\titlebox{5cm}
% You can expand the titlebox if you need extra space
% to show all the authors. Please do not make the titlebox
% smaller than 5cm (the original size); we will check this
% in the camera-ready version and ask you to change it back.

\newcommand\BibTeX{B{\sc ib}\TeX}

\title{Instructions for ACL-2016 Proceedings}

\author{First Author \\
  Affiliation / Address line 1 \\
  Affiliation / Address line 2 \\
  Affiliation / Address line 3 \\
  {\tt email@domain} \\\And
  Second Author \\
  Affiliation / Address line 1 \\
  Affiliation / Address line 2 \\
  Affiliation / Address line 3 \\
  {\tt email@domain} \\}

\date{}

\begin{document}
\maketitle
\begin{abstract}
%  This document contains the instructions for preparing a camera-ready
%  manuscript for the proceedings of ACL-2016. The document itself
%  conforms to its own specifications, and is therefore an example of
%  what your manuscript should look like. These instructions should be
%  used for both papers submitted for review and for final versions of
%  accepted papers.  Authors are asked to conform to all the directions
%  reported in this document.
\end{abstract}


\section{Introduction}
There exists a great deal of textual information in the financial market, such as financial reports and investor message boards.
These soft information probably contains copious signals that enormously affect those observable economic numbers,
like stock prices and the corresponding trading volume.
Due to such close relationship between soft and hard information,
a growing body of accounting and finance research use textual analysis.
For example, academics investigate how financial sentiment keywords in annual SEC-mandated financial reports
influence investors' expectation about a company's future stock prices \cite{loughran2011liability}.\footnote{SEC indicates Securities and Exchange Commission.
One of the SEC-mandated financial reports is 10-K reports.}
Practitioners such as fund managers utilize these sentiment keywords to offer prospects for companies and design their own investment strategies.
However, they face at least two major hurdles when analyzing financial textual information.
First,
the quantity of these textual information is too overwhelmingly large to be systematically analyzed.
Sentiment keywords concerned by market participants are hard to be properly retrieved because they are context-sensitive.
Following this, it is even more challenging to assess and rank the influence of these keywords on target economic numbers.
Second,
given sentiment keywords, the existence and amount of other keywords that have similar impacts on target economic numbers are always questioned.


We have to learn....



\section{Related Work}

\section{System Description}


\section{Case Study}
\subsection{Case 1: Ranking risk levels of companies via keywords in financial reports}
\subsection{Case 2: Financial keyword expansion}
\section{Conclusions and Future Work}

\section*{Acknowledgments}


\bibliography{acl2016}
\bibliographystyle{acl2016}

\appendix
\section{Supplemental Material} \label{sec:supplemental}
\section{Multiple Appendices}


\end{document}
